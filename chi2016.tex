\documentclass[a4paper]{article}

\usepackage{amsmath,amssymb,bm}
\usepackage{xcolor}
\usepackage{booktabs}
\usepackage{floatrow}
\usepackage{blindtext}
\usepackage[english]{babel}
\usepackage[utf8x]{inputenc}
\usepackage{amsfonts}
\usepackage{graphicx}
\usepackage[colorinlistoftodos]{todonotes}

\title{Quantifying Pains and Pleasures in Game, A Reinforcement Learning Approach}
\author{Bx. Wang}

\begin{document}
\maketitle

\section{Related Work}

\subsection{Online Game Player Motivation And Rewards}

One reason that online games appeal so many players is that it provides the environments for different kinds of play styles. The same online game may have respective meanings for different players. In the environment players gain certain kind of satisfactions (rewards) they want by conducting their actions. It's natural to assume a user's behaviors are highly correlated with the user's demanding satisfaction. And for game designers, finding the demanding satisfactions of users could make it easier to serve the users with better experience. 

Previous study categories the satisfactions into three major aspects, namely, achievement satisfactions, social satisfactions, and immersion satisfactions. It was then divided into ten subcategories, briefed in table. ~cite{tbl:satisfactions}. More detailes are introduced in ~\cite{} and ~\cite{}. The weights associated with these ten kinds of satisfactions basically chartered a user's profile, and it would be very meaningful the weights can be recovered in a purely data-driven process, based on the user's playing history.


\begin{tabular}{l|l|p{9cm}}
    \toprule
    Component  &Sub-component & Description \\
    \midrule
    Achievement & Advancement &  \\
     & Mechanics &  \\
     & Competition &  \\
    Social & Socializing &  \\
     & Relationship &  \\
     & Teamwork &  \\
    Immersion & Discovery &  \\
     & Role Playing &  \\
     & Customization &  \\
     & Escapism &  \\
    \bottomrule
    \label{tbe:satisfactions}
\end{tabular}

\subsection{Player Profile Study}



\subsection{Reinforcement Learning and Apprenticeship learning}

Reinforcement learning (RL), especially deep reinforcement learning (DRL), is an [] domain  inspired by behaviorist psychology. In RL, the agent performs in an online environment (in which the agent conducts a sequence of actions and get a reward for each of its actions) and evolves itself from the pains and pleasures it received from its previous actions. In the environment the agent has to deal with the tradeoff between exploration and exploitation, which presents in many real life situations. For example, to level up quickly in a game in a long-term perspective, the player has to conduct some preparation work which does not benefit its leveling up in the near future.

RL is able formulate many online problems to model users' behavior. For example, the trained agent conducts superior play compared with professional human in the game of Go \cite{}, Atari 2600 \cite{}, and Poker \cite{}. Also used to model users' clickings and browsings for online shopping websites, and how users behaves in many optimal control problems etc. (Note: I want an example more related to wowah). It's worth to note that recent advancement of DRL makes it possible to handle complex environment with high-dimension observation and state spaces, making it possible to model a wide variety of real-life problems.

In most cases of user behavior analysis and modeling, the reward scheme of the environment is not explicitly given. For example, the user experience of a software or a online game is the combination of different kinds of satisfication. Instead, apprenticeship learning \cite{}, also known as apprenticeship via inverse reinforcement learning (AIRP), tries to recover the underlying reward function using the observed behavior of the agent, e.g. the trajectories of players in a online game. Applying AIRP on users' behavior log could help the developers understand the intention of the user, the divergence between different group of users, and the dynamic of user profiles. 

\section{Solution}

\subsection{Dataset Description}

To model the user behavior, we treat user as the agent who conduct an action every 10 minutes; during the 10-minute interval, the user decide the zone to stay in the next interval. If the user has been in multiple zones in a single interval, only the zone with longest stay duration was recorded. The actions are represented by zone IDs, ranging from 0 to 164, corresponding to 165 zones existing during Jan 2006 to Jan 2009 in WoW. When counting the index of interval, we ignore those minutes that the user's offline.

We use World of Warcraft Avatar History (WoWAH) dataset \cite{}, a dataset collected from realm \textit{TW-Light's Hope} during 1st Jan 2006 and 10th Jan 2009, containing 70,055 users after we filtering out those with too short playing history. Each user has spent 440.4 time intervals online on average. The dataset contains many different kinds of users: both novice and expert, guild members and isolated players, low-level and high-level players etc., with their respect class, and race in game. The detailed attributes are listed in Table~ref{tbl:attributes}.

During the gameplay the user is able to observe its current game states, including all the attributes recorded in the dataset. We construct the observation vector of the agent using the sequence of attributes extracted from its game plays in the most recent session (since the lastest log on). Instead of using the raw, concatenated vector of those attributes, we employ a preprocess to reduce the redundency, and make those decisive information more explicitly shown to the agent.

To model the user's behavior in a reinforcement learning perspective, we define the reward function which the agent tries to optimize during the gameplay. The reward is separated into five parts $r_1,\cdots r_5$, corresponding to five different kind of satisfactions defined in ~\cite{}. The construction of $r_1,\cdots r_5$ are illustrated in Table \ref{tbl:rewards}. Those rewards are normalized into the same scale.



\begin{tabular}{l|l|p{7cm}}
    \toprule
    Satisfaction & Category & Definition \\
    \midrule
    $r^1$ & Advancement & The speed the user collecting experience and leveling up in game. It's the most common reward a user could receive \\
    $r^2$ & Competition & The satisfaction the user get by joining battleground or arena and competing with human opponents \\
    $r^3$ & Relationship & The long-term relationship with the user's guild, which is quantified by the time elapse since the user join its guild \\
    $r^4$ & Teamwork & The satisfaction the user get by playing in a zone which is featured by teamwork, e.g. Battleground, Arena, Dungeon, Raid, or a zone controlled by The Alliance. \\
    $r^5$ & Escapism & Escapism begins to cumulate if the user has been online for 4 hours or has been regularly login to the game for 20 days. \\
    \bottomrule
    \label{tbl:rewards}
\end{tabular}

\subsection{Reward Weight Recovery}

We apply apprenticeship via inverse reinforcement learning, in order to recover the underlying reward mechanism of the players, using the trajectory $\tau$ of a user (or a group of users). Assume the total reward a user receives is the convex combination of the five rewards listed in Table. \ref{tbl:rewards}. Let $f_t=(r_t^1,r_t^2,r_t^3,r_t^4,r_t^5)$ be the rewards the user received at time $t$, we have the total reward the user receives at time $t$ $$r_t=\phi^Tf_t,$$ 
where $\phi$ is the combination weight with $||\phi||_1=1, \; \phi \geq 0$. Assume at each time step, the user tries to take an action $a$ according to the current game state so as to maximize the best expected cumulative discounted (with discount $\gamma$ over time) reward (known as the action-value function) $Q^\ast(s, a)$, where
$$Q^\ast(s,a)=\mathbf{E}[R_t | s_{t}=s, a_{t}=a | \pi^\ast]$$
and
$$R_t=\sum_{t^\prime\geq t}\gamma^{t^\prime-t}r_{t^\prime}=\sum_{t^\prime\geq t}\gamma^{t^\prime-t}\phi^Tf_{t^\prime}.$$
The term $\pi^\ast$ indicates optimal policy, described by a distribution $\mathbb{P}(a|s)$ over the feasible action space $\mathcal{A}(s)$. In this setting, the weight $\phi$ must satisfy that the action the user has taken must induce a larger $Q^\ast$ value than any other valid action would have done. This optimality infers that
\begin{equation}
Q^\ast(s,a) \geq \max_{a^\prime \in A(s)}Q^\ast(s,a^\prime) \label{eqn:irl}
\end{equation}
is satisfied for all $(s,a)$ pairs appeared in the user's trajectory. Consider the existence of possible sub-optimal actions conducted by the user, we introduce slack variables $\xi_{s,a}$ into the problem formulation. Let $\xi_{s,a}$ be the difference of the actual action-value $Q^\ast(s,a)$, and the largest possible action-value $\max_{a^\prime \in A(s)}Q^\ast(s,a^\prime)$ whenever Eq. \eqref{eqn:irl} is not satisfied, and zero otherwise. We minimize the summation of $\xi_{s,a}$ over the recorded user's trajectory
$$-C\sum_{s,a} \big[\min(0, Q^\ast(s,a)) - \max_{a^\prime \in A(s)\backslash a}Q^\ast(s,a^\prime)\big],$$
which is then formulated into the following linear programming (LP) problem

\begin{equation*}
\begin{aligned}
& \underset{\phi^i, \xi_{s,a}}{\text{minimize}}
& & C\sum \xi_{s,a} \\
& \text{subject to}
& & \phi^T(Q(s,a)-Q(s,a^\prime)) \geq - \xi_{s,a}, \; \forall (s,a) \in \tau, a^\prime \in \mathcal{A}(s)\backslash a \\
&&& \phi \geq 0 \; \\
&&& \xi_{s,a} \geq 0 \; \forall (s,a) \in \tau.
\label{eqn:lp}
\end{aligned}
\end{equation*}


\subsection{Action-Value Function Approximation}

To solve \ref{eqn:lp} we have to be able to evaluate the $Q^*(s,a)$ value for every $(s,a)$ pair. Let
$$Q^i(s,a)=\mathbf{E}[\sum_{t^\prime\geq t}\gamma^{t^\prime-t}r^i_{t^\prime} | s_{t}=s, a_{t}=a | \pi^{i,\ast}] \label{eqn:qi}$$
be the action-value function, when the user only takes the $r^i$ into account and ignores all four other kinds of satisfactions. By definition we have $Q^*(s,a)=\phi^TQ(s,a)$, and it becomes suffice to the following linear programming, if $Q^i(s,a)$ can be evaluated for all $i$ and $(s,a)$ pairs in the trajectory.

It suffices to solve $Q^i(s,a)$. As the number of feasible states $s$ could be arbitrary large, for an user in WoW, it's impossible to enumerate over the state space. Instead, we use deep-Q networks (DQN) \cite{} to approximate the $Q^i$ functions. The neural networks takes $s$ as input, and output the $Q$ value for every action $a$. We use the same network architecture for $i=1,\cdot,5$, illustrated in Fig. \ref{fig:arch}. The categorical elements in $s$ are firstly processed by an embedding layer, while the elements with real values are fed into an fully connected (FC) layer with rectifier non-linearity. The output of embedding layer and FC layer are then concatenated and fed into another FC layer with rectifier nonlinearity. A final FC layer is applied to compute the $Q$ value for each action $a^\prime$.

Denote the trainable parameters in the Q-network as $\theta^i$, we optimize over $\theta^i$ in order to approximate Eq. \eqref{eqn:qi}. The key observation of Q-learning is that, the action-value function, by its definition, should satisfy the Bellman equation \cite{}. That is, if the user takes action $a$ and the state turns into $s_{t+1}$ from $s_t$, we have
$$Q^i(s_t,a_t)=r_{t} + \gamma \max_{a^\prime}Q^i(s_{t+1}, a^\prime). \label{eqn:bellman}$$
Q-learning tries to find the action-value function that satisfies Eq. \eqref{eqn:bellman}, by minimizing the squared difference $L$ between both sides of the equation.
$$L^i=\mathbb{E}_{s_t, a_t, r_t, s_{t+1}} \frac{1}{2}(Q^i(s,a)- r_{t} + \gamma\max_{a^\prime}Q(s_{t+1}, a^\prime))^2.$$
The function $Q^i$, and thus $L$, are differentiable with respect to $\theta^i$. Hence $\theta^i$ can be updated with stochastic gradient descent, by
\begin{eqnarray*}
\theta^i & \leftarrow & \theta^i - \alpha\frac{\partial L}{\partial \theta^i} \\
\end{eqnarray*}
We take advantage of the algorithm introduced in \cite{}, that the target network only getting updated periodically to reduce its correlation with the Q-network, which is important for the stability of DQN. The derivative of $L$ with respective $\theta^i$ becomes
$$\frac{\partial L}{\partial \theta^i} = \mathbb{E}_{s_t, a_t, r_t, s_{t+1}} (Q^i(s,a)- r_{t} + \gamma\max_{a^\prime}Q(s_{t+1}, a^\prime | \theta^{i -}))\cdot \frac{\partial{Q^i(s,a)}}{\partial{\theta^i}},$$
where $\theta^{i-}$ is the network parameter who is assigned current $\theta^i$ value periodically during training.

\subsection{Agreement between $Q^1$ and Expert Behaviors}

We conduct a quantitative evaluation of our learned $Q^1$ network. Consider collecting experience and getting level up is one of the major objective for players, we evaluate if the actions predicted by $Q^1$ agree with the actual play conducted by those experts (who level up quickly). At state $s$, the predicted action is the one with the largest action-value $Q^1(s,a)$, i.e.,
$$a = \text{argmax}_{a^\prime}Q^1(s,a^\prime).$$
For the $(s,a)$ pairs extracted from the top 200 (in leveling up speed) players' trajectories, the prediction accuracy is 0.36 with the total number of feasible actions $|\cup_s\mathcal{A}(s)|=156$, corresponding to 156 different zones in WoW existed during Jan 2006 - Jan 2009.

[take an example here/conclude Q quality]

[compare with supervised learning/linear Q]




\end{document}
